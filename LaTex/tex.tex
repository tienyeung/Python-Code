\documentclass[UTF8]{ctexart}
\usepackage{amsmath}
\usepackage{graphicx}
%版面
\usepackage{geometry}
\geometry{left=1cm,right=2cm,top=3cm,bottom=4cm}
%页眉页脚
\usepackage{fancyhdr}
%间距
\usepackage{setspace}
\onehalfspacing%行
\addtolength{\parskip}{.4em}%段


\title{Learn \LaTeX}
\author{yeung}
\date{\today}

\pagestyle{fancy}
\lhead{\author}
\chead{\date}
\rhead{\LaTeX}
\lfoot{}
\cfoot{\thepage}
\rfoot{}
\renewcommand{\headrulewidth}{0.4pt}
\renewcommand{\headwidth}{\textwidth}
\renewcommand{\footrulewidth}{0pt}

\begin{document}
\maketitle
\tableofcontents
\newpage
\section{hello,china}
china is a great country!

and it's beautiful ,too!
\subsection{hello,beijing}
beijing is a good province!
\subsubsection{hello,chaoyang}
\paragraph{chaoyang}
people are great!
\subparagraph{chaoyang}
people are happy!
\subsection{hello,sichuan}
\paragraph{sichuan}是天府之国
\section{equation}
\subsection{inline}
这是一个$F=ma$公式,这\[E=mc^2\]也是。瞧瞧这个$\sqrt{x}$ 以及\[\frac{1}{2}\]
\subsection{display}
%有编号的公式
\begin{equation}
z=r\cdot e^{2\pi i}
\end{equation}
\begin{equation}
    E=hv
\end{equation}
\subsection{大型公式}
\[\sum_{i=1}^n i \prod_{i=1}^n \]
\[\sum\limits _{i=1}^n i\quad \prod\limits _{i=1}^n \]
\[lim_{x\to0}x^2 \quad \int_a^b x^2 dx \]
\subsection{省略号}
\[ x_1,x_2,\dots ,x_n\quad 1,2,\cdots ,n\quad
\vdots\quad \ddots \]
\subsection{括号}
$\Bigl(\bigl((x)\bigr)\Bigr)$
\subsection{矩阵}
\[ \begin{pmatrix} a&b\\c&d \end{pmatrix} \quad
\begin{bmatrix} a&b\\c&d \end{bmatrix} \quad
\begin{Bmatrix} a&b\\c&d \end{Bmatrix} \quad
\begin{vmatrix} a&b\\c&d \end{vmatrix} \quad
\begin{Vmatrix} a&b\\c&d \end{Vmatrix} \]
and I have a little matrix here:$(\begin{smallmatrix} a&b\\c&d 
\end{smallmatrix}) $
\subsection{多行公式}
%公式组
\begin{gather}
    a = b+c+d \\
    x = y+z
\end{gather}
\begin{align}
    a &= b+c+d \\
    x &= y+z
\end{align}
%分段函数
\[ y= \begin{cases}
    -x,\quad x\leq 0 \\
    x,\quad x>0
\end{cases} \]
\section{photo and table}
\subsection{photo}
\paragraph{like this}
\includegraphics[width=.8\textwidth]{cat.jpg}
\subsection{table}
\begin{tabular}{||l|c|r||}
    \hline
   操作系统& 发行版& 编辑器\\
    \hline\hline
   Windows & MikTeX & TexMakerX \\
    \hline
   Unix/Linux & teTeX & Kile \\
    \hline
   Mac OS & MacTeX & TeXShop \\
    \hline
   通用& TeX Live & TeXworks \\
    \hline
\end{tabular}
\subsection{float}
%htbp:here,top,bottom,float page
\begin{figure}[h]
    \centering
    \includegraphics[width=.8\textwidth]{cat.jpg}
    \caption{小猫猫}
    \label{fig:cutecat}
\end{figure}
\end{document}